\documentclass{article}

\usepackage[utf8]{inputenc}
\usepackage[T1]{fontenc}
\usepackage{lmodern}
\usepackage[ngerman]{babel}
\usepackage{amsmath}
\usepackage{amsfonts}
\usepackage{graphicx}
\usepackage{hyperref}
\usepackage{media9}
\usepackage{multimedia}
\usepackage{marvosym}
\usepackage{enumitem,xcolor}

\title{Feststellungen zu Anchored Rectangle Packing Implementationen}
\author{Lars Müller}
\date{27. - 31. Januar 2020}
\begin{document}

\maketitle
\tableofcontents
\section{Einleitung}


\subsection{Definitionen}

\begin{itemize}
    \item ``GreedyPacking'': Wie im Paper von Dumitrescu und Toth
    \item ``FastGreedyPacking'': Ähnliche Sweepline (entgegengesetzt), anderer ``Unterschritt''
    \item ``ImprovePacking'': Verbessert Packings mit Gradient Descent
    \item ``BrutePacking'': Brute-Force - Ausprobieren aller Packing-Möglichkeiten
    \item ``BrutyPacking'': Ausprobieren aller möglichen Iterationsreihenfolgen
\end{itemize}

\subsection{Feststellungen}

\begin{itemize}
    \item ``BrutePacking'' und ``BrutyPacking'' nicht zielführend, da viel zu langsam
    \item ``FastGreedyPacking'' schneller als ``GreedyPacking''
    \item ``GreedyPacking'' liefert meist (aber nicht immer) bessere Ergebnisse
    \item ``ImprovePacking'' + ``FastGreedyPacking'' liefert meist bessere Ergebnisse als ``GreedyPacking''
    \item ``ImprovePacking'' + ``GreedyPacking'' liefert noch bessere Ergebnisse, ist aber noch langsamer
\end{itemize}

\end{document}